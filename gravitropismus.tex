\documentclass[
a4paper, 
11pt, 
ngerman,
listof=totoc,
%bibliography=totoc,
bibliography=totocnumbered
]{scrreprt}


\usepackage[T1]{fontenc}
\usepackage[utf8]{inputenc}
\usepackage[ngerman]{babel}

\usepackage[
backend=biber,
style=authoryear-ibid,
%sorting=ynt
]{biblatex}
\addbibresource{gravitropismus-bibliography.bib}

\title{Untersuchung von Gravitropismus bei \emph{Lepidium sativum} mit einem selbstgebautem Klinostat unter Zimmerbedingungen}

\subtitle{W-Seminararbeit im Fach Biologie am Luitpold-Gymnasium München}

\begin{document}
\maketitle
\tableofcontents

\chapter{Einleitung}

Vorschläge für Überschrift statt {\glqq Einleitung\grqq} 

Zitatbeispiel \parencite[29]{campbell}

\begin{itemize}
	\item Bedeutung von Gravitropismus in der Biologie
	\item Gravitropismus als wichtige Pflanzeneigenschaft 
\end{itemize}

\chapter{Fachliche Analyse der Thematik: Gravitropismus}

\chapter{Experimenteller Nachweis von Gravitropismus bei \emph{Lepidium sativum}}

\section{Methoden}

\section{Ergebnisse}

\section{Diskussion}

%\chapter{Schluss}

\printbibliography



\end{document}
