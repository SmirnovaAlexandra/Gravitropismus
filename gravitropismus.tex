\documentclass[
a4paper, 
11pt, 
ngerman,
listof=totoc,
%bibliography=totoc,
bibliography=totocnumbered
]{scrreprt}


\usepackage[T1]{fontenc}
\usepackage[utf8]{inputenc}
\usepackage[ngerman]{babel}
\usepackage{graphicx}
\usepackage{lipsum}

\usepackage[
backend=biber,
style=authoryear-ibid,
%sorting=ynt
]{biblatex}
\addbibresource{gravitropismus-bibliography.bib}

\title{Untersuchung von Gravitropismus bei \emph{Lepidium sativum} mit einem selbstgebautem Klinostat unter Zimmerbedingungen}

\subtitle{W-Seminararbeit im Fach Biologie am Luitpold-Gymnasium München}

\begin{document}
\maketitle
\tableofcontents

\chapter{Gravitropismus als wichtige Pflanzeneigenschaft}
Pflanzen nehmen einen wichtigen Platz in unserer Welt ein.Sie geben uns nicht nur den Sauerstoff zum Atmen, sondern auch eine lebenswichtige Ernährung sowie weiteren Nutzen in der Medizin und Kosmetik. Aber was würden Pflanzen uns bringen, wenn sie in der Umwelt nicht überleben würden? Wenn sie bei jedem Sturm oder Druck knicken und dann verwelken würden? Nichts, und deshalb spielen Pflanzeneigenschaften wie zum Beispiel Gravitropismus eine wichtige Rolle. Denn ohne den Gravitropismus wäre es nicht möglich beispielsweise Ackerbau zu betreiben. Denn nur der Gravitropismus sorgt für eine aufrechte stabile Haltung in fast jedem Ort. Und dieser Eigenschaft ist diese Arbeit gewidmet. 

\chapter{Fachliche Analyse der Thematik: Gravitropismus}

\section{Gravitropismus}
Die Bestimmung der Wachstumrichtung der Wurzel und des Sprosses unter dem Einfluss der Schwerkraft wird als Gravitropismus (Geotropismus) bezeichnet. Dabei werden drei Bewegungen unterschieden: Positiv gravitrop, negativ gravitrop und transversalgravitrop. Positiv gravitrop bedeutet, dass das Wachstum zur Schwerkraftquelle hin (nach unten) erfolgt. Positiv gravitrope Organe wären zum Beispiel Wurzeln, Rhizoide (wurzelähnliche Strukturen), Moose oder Farnprothallien. Negativ gravitrop dagegen bedeutet, dass die Organe wie Sprossen, Sporangienträger der Schimmelpilze der Gattung Mucor oder Fruchtkörper mancher Pilze von der Schwerkraftquelle weg (nach oben) wachsen. Seitenwurzeln der ersten Ordnung (Nebenwurzeln, die von der Hauptwurzel entspringen) und zahlreiche Seitenzweige sowie Blätter wachsen transversalgravitrop: entweder flach oder quer nach unten in einem speziellen Winkel 
\parencite[449]{Strasburger}. 
Legt man eine Pflanze quer, so werden sich die Organe, Wurzeln und Sproß, krümmen, bis sie senkrecht stehen und wieder positiv bzw. negativ gravitrop wachsen
\parencite[528]{Luettge}.

\section{Differenzielles Wachstum}


\section{Koordination von Gravitropismus durch Pflanzenhormone}

\chapter{Experimenteller Nachweis von Gravitropismus bei \emph{Lepidium sativum}}

\section{Methoden}

\section{Ergebnisse}

\section{Diskussion}

\chapter{Fazit und Ausblick}


\printbibliography

% afa
% \begin{figure}
% 	\centering
% 	\includegraphics[width = .5\linewidth]{images/IMG_1117.JPG}
% 	\caption{a nice little caption \label{nice_picture}}
% \end{figure}
% asd
% a
% 
% as we see in image \ref{nice_picture}
% 
% asfda
% 
% asd

\end{document}
