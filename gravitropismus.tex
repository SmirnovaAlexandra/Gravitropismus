\documentclass[
a4paper, 
11pt, 
ngerman,
listof=totoc,
%bibliography=totoc,
bibliography=totocnumbered
]{scrreprt}


\usepackage[T1]{fontenc}
\usepackage[utf8]{inputenc}
\usepackage[ngerman]{babel}

\usepackage[
backend=biber,
style=authoryear-ibid,
%sorting=ynt
]{biblatex}
\addbibresource{gravitropismus-bibliography.bib}

\title{Untersuchung von Gravitropismus bei \emph{Lepidium sativum} mit einem selbstgebautem Klinostat unter Zimmerbedingungen}

\subtitle{W-Seminararbeit im Fach Biologie am Luitpold-Gymnasium München}

\begin{document}
\maketitle
\tableofcontents

\chapter{Gravitropismus als wichtige Pflanzeneigenschaft}

Vorschläge für Überschrift statt {\glqq Einleitung\grqq} 

Zitatbeispiel \parencite[29]{campbell}


\begin{itemize}
	\item Bedeutung von Gravitropismus in der Biologie

\end{itemize}

\chapter{Fachliche Analyse der Thematik: Gravitropismus}

\section{Gravitropismus}
Die Bestimmung der Wachstumrichtung der Wurzel und des Sprosses unter dem Einfluss der Schwerkraft wird als Gravitropismus (Geotropismus) bezeichnet. Dabei werden drei Bewegungen unterschieden: Positiv gravitrop, negativ gravitrop und transversalgravitrop. Positiv gravitrop bedeutet, dass das Wachstum zur Schwerkraftquelle hin (nach unten) erfolgt. Positiv gravitrope Organe wären zum Beispiel Wurzeln, Rhizoide (wurzelähnliche Strukturen), Moose oder Farnprothallien. Negativ gravitrop dagegen bedeutet, dass die Organe wie Sprossen, Sporangienträger der Schimmelpilze der Gattung Mucor oder Fruchtkörper mancher Pilze von der Schwerkraftquelle weg (nach oben) wachsen. Seitenwurzeln der ersten Ordnung (Nebenwurzeln, die von der Hauptwurzel entspringen) und zahlreiche Seitenzweige sowie Blätter wachsen transversalgravitrop: entweder flach oder quer nach unten in einem speziellen Winkel. 
\parencite[449]{Strasburger1991} ? 

\section{Differenzielles Wachstum}

\section{Koordination von Gravitropismus durch Pflanzenhormone}

\chapter{Experimenteller Nachweis von Gravitropismus bei \emph{Lepidium sativum}}

\section{Methoden}

\section{Ergebnisse}

\section{Diskussion}

\chapter{Fazit und Ausblick}


\printbibliography



\end{document}
