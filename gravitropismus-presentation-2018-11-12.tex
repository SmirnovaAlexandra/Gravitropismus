\documentclass[aspectratio=169]{beamer}
\usetheme{default}

\usepackage[utf8]{inputenc}
\usepackage[ngerman]{babel}
\usepackage{graphicx}
\usepackage{csquotes}

\usepackage{lmodern}

%\usepackage{scrextend}
%\changefontsizes{18pt}

\usepackage[
backend=biber,
style=authoryear-ibid,
%sorting=ynt
]{biblatex}
\addbibresource{gravitropismus-bibliography.bib}

\author{Alexandra Smirnova}

\title{Präsentation zur Seminararbeit \hyphenquote{ngerman}{Gravitropismus}}
\subtitle{W-Seminar Biologie}

%\title{\hyphenquote{ngerman}{Gravitropismus}}
%\subtitle{Präsentation zum W-Seminar Biologie}

%\logo{}

%\institute{}

\date{19. Dezember 2018}

%\subject{}

\setbeamercovered{transparent}
%\setbeamertemplate{navigation symbols}{}
\setbeamertemplate{section in toc}[sections numbered]
\setbeamertemplate{subsection in toc}[subsections numbered]
\setbeamertemplate{caption}[numbered]


\useoutertheme{infolines}

\begin{document}
	\maketitle
	
	\begin{frame}
		\frametitle{Gliederung}
		\tableofcontents
	\end{frame}
	
	\section{Grundlagen von Gravitropismus}

	
	\begin{frame}
		\frametitle{Grundlagen von Gravitropismus}
	\end{frame}
	
	\subsection{Arten von Gravitropismus}
	
	\begin{frame}
		\frametitle{Arten von Gravitropismus}
		Positiv gravitrop - zur Schwerkraftquelle hin (nach unten zur Erdmitte)
		
		Negativ gravitrop - von der Schwerkraftquelle entgegengesetzt (nach oben)
		
		Transversalgravitrop - entweder horizontal oder quer nach unten in einem bestimmten Winkel 
		
	\end{frame}
	
	\subsection{Prozess der gravitropischen Krümmung}
	
	\begin{frame}
		\frametitle{Prozess der gravitropischen Krümmung}
	\end{frame}
		
	\subsubsection{Reizaufnahme bei Pflanzen}
		
	\begin{frame}
		\frametitle{Reizaufnahme bei Pflanzen}
	\end{frame}
			
	\subsubsection{Signaltransduktion}
		
	\begin{frame}
		\frametitle{Signaltransduktion}
	\end{frame}
			
	\subsubsection{Differenzielles Wachstum}
		
	\begin{frame}
		\frametitle{Differenzielles Wachstum}
	\end{frame}
	
	\section{Experimenteller Nachweis von Gravitropismus bei \protect\emph{Lepidium sativum}}
	
	\begin{frame}
		\frametitle{Experimenteller Nachweis von Gravitropismus bei \protect\emph{Lepidium sativum}}
	\end{frame}	
	
	\subsection{Methoden}
	
	\begin{frame}
		\frametitle{Methoden}
	\end{frame}
	
	\subsubsection{Pflanzen, Material und Geräte}

	\begin{frame}
		\frametitle{Pflanzen, Material und Geräte}
	\end{frame}
	
	\subsubsection{Versuchsmethodik}
	
	\begin{frame}
		\frametitle{Versuchsmethodik}
	\end{frame}
	
	\subsection{Durchführung und Ergebnisse}
	
	\begin{frame}
		\frametitle{Durchführung und Ergebnisse}
	\end{frame}
	
	\subsubsection{Vorbereitung, Ankeimen}
	
	\begin{frame}
		\frametitle{Vorbereitung, Ankeimen}
	\end{frame}
	
	\subsubsection{Klinostat-Experiment}
	
	\begin{frame}
		\frametitle{Klinostat-Experiment}
	\end{frame}
	
	\subsubsection{Ausrichtungs-Experiment}
	
	\begin{frame}
		\frametitle{Ausrichtungs-Experiment}
	\end{frame}
	
	\subsection{Diskussion und Fazit}
	
	\begin{frame}
		\frametitle{Diskussion und Fazit}
	\end{frame}

\end{document}